\documentclass[pdftex,a4paper]{scrbook}

% \usepackage{ngerman}
\usepackage[latin1]{inputenc}
\usepackage[T1]{fontenc}




\begin{document}

\title{Sparse Coding}
\subtitle{...progress report}
\author{Sebastian Szczepanski}
\institute[TU Berlin]{TU Berlin Computer Graphics}
\date{\today}

\begin{frame}
\titlepage
\end{frame}

\begin{frame}
\frametitle{Outline}
\tableofcontents[part=1,pausesections]
\end{frame}

\begin{frame}
\frametitle{Titel}
Inhalt
\begin{block}{Problem}
"Sparse coding was originally developed for studying how neurons in the brain responded to visuals. It works by breaking down an image—for simplicity's sake, usually one in grayscale—into mathematical functions, pixel by pixel. The images that are broken down are just small patches of whole works, not much more than a dozen pixels square."
\end{block}
\begin{itemize}
\item Punkt 1
\item 2. Punkt
\item dritter Punkt
\end{itemize}
\end{frame}
\section{motivation}
\subsection{advantages vs. fft/wavelet}
\subsection{the problem}
\begin{frame}
\[ 
\min_{\alpha\in} \frac{1}{2} \lVert x - D\alpha \rVert^{2}_{2} + \psi(\alpha)
\] 
\end{frame}
\subsection{new discoveries}
\section{sparse coding}
\subsection{basics}
\subsection{optimization}
\subsection{results}
\subsection{problems}

\end{document}