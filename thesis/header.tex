\linespread{1.5}
% \usepackage{setspace}
% \onehalfspacing

\usepackage[english]{babel}
\usepackage[utf8]{inputenc}
\usepackage{lmodern}            % use lmodern fonts
%\usepackage{times}
%\usepackage{dsfont}
\usepackage[T1]{fontenc}
\setcounter{secnumdepth}{3}     % limit enumeration depth
\setcounter{tocdepth}{3}        % limit TOC depth
\usepackage{varioref}           % nice refs
\usepackage{amsmath}            % math fonts
\usepackage{amssymb}            % math symbols
\usepackage{units}   
\usepackage{xcolor}
\usepackage{graphicx}
\usepackage{todo}
\usepackage[pdftex]{hyperref}
\usepackage{cite}
\usepackage{url}

%\usepackage[clearempty]{titlesec}
%\usepackage{tikz}

\usepackage{prettyref}
\newrefformat{eq}{Eqn.~\textup{(\ref{#1})}}
\newrefformat{tab}{Table~\ref{#1}}
\newrefformat{alg}{Algorithm~\ref{#1}}
\newrefformat{fig}{Figure~\ref{#1}}
\newrefformat{chp}{Chapter~\ref{#1}}


% Olga's matrix and vector definitions
\renewcommand{\vec}[1]{\boldsymbol{\mathrm{#1}}}
\newcommand{\mat}[1]{\boldsymbol{\mathrm{#1}}}
\let\set=\mathcal
\let\func=\mathrm

% Create a nice looking argmin operator with a
% small space between arg and min
\DeclareMathOperator*{\argmin}{arg\,min}
\DeclareMathOperator*{\argmax}{arg\,max}

% ALGORITHMS
\usepackage[chapter]{algorithm}
\usepackage{algorithmic}
%\numberwithin{algorithm}{chapter} 

% LISTINGS
\usepackage{listings}
\definecolor{body}{rgb}{.8,.9,1.0}
\definecolor{keyword}{rgb}{0.5,0.5,0}
\lstloadlanguages{C,C++,Matlab}
\lstset{
%  language=C++,
  frame=tb,
  numbers=left,
  numberstyle=\tiny,
%  basicstyle=\small,
  basicstyle=\scriptsize,
%  basicstyle=\footnotesize,
%  backgroundcolor=\color{body},
%  keywordstyle=\color{keyword},
  commentstyle=\color{blue},
  stringstyle=\color{green},  
  morecomment=[s][\color{blue}]{/*}{*/}
}

% use nice footnote indentation
\deffootnote[1em]{1em}{1em}{\textsuperscript{\thefootnotemark}\,}

\usepackage[Lenny]{fncychap}
% page header & footer
\usepackage{fancyhdr}
\pagestyle{fancy}
\fancyhf{}
\fancyhead[EL]{\thepage}% gerade Seiten, links
\fancyhead[ER]{\leftmark}% gerade Seiten, rechts
\fancyhead[OL]{\rightmark}% ungerade Seiten, links
\fancyhead[OR]{\thepage}% ungerade Seiten, rechts

\fancypagestyle{plain}{%
\fancyhf{}
\fancyhead[EL]{\thepage}% gerade Seiten, links
\fancyhead[ER]{\leftmark}% gerade Seiten, rechts
\fancyhead[OL]{\rightmark}% ungerade Seiten, links
\fancyhead[OR]{\thepage}% ungerade Seiten, rechts

}
\renewcommand{\sectionmark}[1]{
\markboth{\thesection{} #1}{\thesection{} #1}
}
\renewcommand{\subsectionmark}[1]{
\markright{\thesubsection{} #1}
}






