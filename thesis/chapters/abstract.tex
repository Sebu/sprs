\newpage
\phantomsection
\addcontentsline{toc}{chapter}{Zusammenfassung}
\chapter*{Zusammenfassung}
\thispagestyle{empty}

Bilder koennen aus wenige fundamentale Basislementen zusammengesetzt werden.
Eine Repr\"asentationsform die bei Bildoperationen wie dem
Enfernen von Rauschen, Kompression oder Klassifikation genutzt wird.
%Diese Vorgehensweise ist auch als sparse coding bekannt.
Sammlungen solcher Basiselemente zu lernen anstatt sie zu konstruieren kann
zu Qualit\"{a}tsverbesserungen bei Bildoperationen f\"{u}hren. 
In den letzten Jahrzehnten wurden mehrere Algorithmen vorgestellt um solche 
Basiselemente zu lernen.

Im Zuge dieser Diplomarbeit wurde ein Software entwickelt die einen neuen
Lernalgorithmus nutzt um Basiselemente f\"ur grosse Bilddatenbanken
zu lernen. Eine Herrausforderung dabei ist es Beschr\"{a}kungen des
Lernprozesses bez\"{u}glich Geschwindigkeit und Konvergenz zu bew\"altigen. Ein
Clusteransatz wird eingesetzt um diese Probleme zu bewerkstelligen.

Die Software wird genutzt um zu untersuchen wie gute
Basiselemente f\"{u}r grossen Mengen von Bilder lernen werden k\"{o}nnen.
Ein Beispielanwendung gezeigt wie die gelernten Basiselemente in einem
Kompressionverfahren eingesetzt werden k\"{o}nnen das mit anderen Verfahren
mithalten kann.

\newpage
\phantomsection
\addcontentsline{toc}{chapter}{Abstract}
\chapter*{Abstract}
\thispagestyle{empty}

Images can be composed from few fundamental basis elements. 
A representation used by image operations such as noise removal,
compression or classification. Learning collections of those basis elements
rather than designing them can lead to quality improvements of those
operations. In the last decade several algorithms for learning such basis
elements were proposed.

This thesis develops a framework that uses a online learning
algorithms to learn basis elements that can be used for large image
databases. One challenge is to overcome speed and convergence limitations of the
learning algorithm. A clustering approach is applied To address these issues. 

The framework is used to explore ways to learn good basis elements for
large sets of images. An example application shows that the learned elements
can be used in a compression algorithm that can lead to similar quality as other
algorithms.


% Images can be decomposed into few fundamental basis elements. 
% This procedure is also known as sparse coding. Learning collections of
% those basis elements rather than designing them can lead to quality
%improvements
% of operations on images. Such as noise removal, compression or classification.
% In the last decade several algorithms for learning such basis elements were
% proposed.
% 
% This thesis develops a framework that uses a new online learning
% algorithms to learn big basis elements that can be used for large image
% databases in tasks such as reconstruction and compression. One challenge is to
% overcome speed and convergence limitation of the learning algorithm. To
%address % these
% issues a clustering approach is applied. 
% 
% The framework is then used to learn big dictionaries from
% large sets of images . The learned dictionaries are used in a new dictionary
% based compression algorithm that yields similar quality as JPEG.
%#. can surpass it.

\phantomsection


