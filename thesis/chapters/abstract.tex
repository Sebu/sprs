\newpage
\phantomsection
\addcontentsline{toc}{chapter}{Zusammenfassung}
\chapter*{Zusammenfassung}
\thispagestyle{empty}

Bilder k\"{o}nnen aus wenigen fundamentalen Basiselementen zusammengesetzt
werden, eine Repr\"asentationsform, die bei Bildoperationen wie dem
Entfernen von Rauschen, Kompression oder Klassifikation genutzt wird.
%Diese Vorgehensweise ist auch als sparse coding bekannt.
Gelernte anstatt konstruierter Basiselemente f\"{u}r solche Bildoperationen zu
nutzen kann dabei zu besseren Ergebnissen f\"{u}hren. In den letzten Jahrzehnten
sind verschiedene Algorithmen zum Erlernen von Basiselementen vorgestellt
worden.

Im Zuge dieser Diplomarbeit wurde eine Software entwickelt die einen neuen
Lernalgorithmus nutzt, um Basiselemente f\"{u}r große Bilddatenbanken
zu lernen. Eine Herausforderung bestand darin, Beschr\"{a}nkungen hinsichtlich
Geschwindigkeit und Konvergenz zu überwinden. Dazu erwies sich ein
Cluster-Ansatz als geeignet.

Die Software wird genutzt um zu untersuchen, wie geeignete Basiselemente
insbesondere für große Bildmengen gelernt werden können. Eine Beispielanwendung
zeigt, dass die so erzeugten Basiselemente in einem Kompressionsverfahren
eingesetzt werden k\"{o}nnen, das anderen Verfahren ebenb\"{u}rtig ist. 

\newpage
\phantomsection
\addcontentsline{toc}{chapter}{Abstract}
\chapter*{Abstract}
\thispagestyle{empty}

Images can be composed from few fundamental basis elements. 
A representation used by image operations such as noise removal,
compression or classification. Using learned rather than designed basis elements
for such operations can lead to better results. In the last decades several
algorithms for learning basis elements were proposed.

This thesis develops a framework that uses an online learning
algorithm to learn basis elements that can be used for large image
databases. One challenge is to overcome speed and convergence limitations of the
learning algorithm. A clustering approach is applied to address these issues. 

The framework is used to explore ways to learn good basis elements for
large sets of images. An example application shows that the learned elements
can be used in a compression algorithm that can compete with other algorithms.

%lead to similar quality as other algorithms.


% Images can be decomposed into few fundamental basis elements. 
% This procedure is also known as sparse coding. Learning collections of
% those basis elements rather than designing them can lead to quality
%improvements
% of operations on images. Such as noise removal, compression or classification.
% In the last decade several algorithms for learning such basis elements were
% proposed.
% 
% This thesis develops a framework that uses a new online learning
% algorithms to learn big basis elements that can be used for large image
% databases in tasks such as reconstruction and compression. One challenge is to
% overcome speed and convergence limitation of the learning algorithm. To
%address % these
% issues a clustering approach is applied. 
% 
% The framework is then used to learn big dictionaries from
% large sets of images . The learned dictionaries are used in a new dictionary
% based compression algorithm that yields similar quality as JPEG.
%#. can surpass it.

\phantomsection


