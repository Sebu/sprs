\newpage
\phantomsection
\addcontentsline{toc}{chapter}{Zusammenfassung}
\chapter*{Zusammenfassung}
\thispagestyle{empty}

\newpage
\phantomsection
\addcontentsline{toc}{chapter}{Abstract}
\chapter*{Abstract}
\thispagestyle{empty}

%Learning the most significant basis elements, that make up an
%signal, is a popular task in signal processing.
Images can be decomposed into a sparse linear combination of basis
elements. This process is known as sparse coding. Learning dictionaries of
those basis elements rather than desiging them can lead to quality improvements
of operations on images. Such as noise removal, classification or compression.
In the last decade several algorithms for learning such basis elements were
proposed.

This thesis develops a framework that uses those sparse learning algorithms to
learn big dictionaries that can be used for large image databases in tasks
such as reconstruction and compression. One challenge is to overcome speed and
convergence limitation of the learning algorithm. To address this issues a
clustering approach is applied. 

For evaluation, the framework is then used to learn big dictionaries from a
large set of images. The learned dictionaries are then tested on reconstruction
tasks on images from the large image database and a simple compression algorihm
is compared to JPEG and JPEG 2000 compression strategies.

\phantomsection


