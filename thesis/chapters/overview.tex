\chapter{Overview}


\section{Problem}
Signal representation via linear combination of <> .
\section{Image bases}
\subsection{basis transformation}

\subsection{sparse coding}
Solve a under-determined linear system of equations. 
Da=X
<>
In the last 15 years several sparse coding algorithms have been proposed. Some that solve the initial problem <> greedyly, the (orthogonal) matching pursuit, and others which modified the problem to become convex/linear. These primary derive from the numerical domain in the form of 
large linear system solvers with few optimization constraints. The LARS-Lasso, basis pursuit, FOCUSS?


blaa \footnote{test 123} blub


\section{Dictionaries}
\subsection{analytical}
\subsection{learned and over-complete}
In the last decade several learning algorithms have been proposed which try to a universal basis that 
can sparsly reconstruct a set of "trainig data" with minimal error. 
K-SVD
MOD
Online learning
Mairal2010

\subsection{Learning for the Task}
It has been shown that learning basis specicial for certain tasks can lead to the best results\cite{}.  <>
Based on this discovery will also concentrate on a specific class. <> Join the basis for natural images and cartoon/line images.
We will also concentrate on real pratical data. This means typically 3-channel data of 1+ megapixel images found on image hosting services like flickr,<>,<>.

\section{Image database}

\section{Related work}
- multi-scale
- multi-channel
- hirarchy
- 

- training with a neural network
Learning Multiple Layers of Features from Tiny Images, Alex Krizhevsky, 2009.
\url{http://www.cs.toronto.edu/~kriz/learning-features-2009-TR.pdf}

\section{Goal}
