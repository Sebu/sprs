\chapter{Overview}
\thispagestyle{empty}
\numberwithin{equation}{chapter}

\section{Dictionaries for signal representation}
%\subsection{Problem statement}

%copy
% When a signal is said to be sparse in an engineering sense, it really means that the signal can be expanded in either
%a small number of terms or in a series with significantly decaying coefficients. In the former case, one talks about 
%a strictly sparse signal, in the latter case, one talks about a compressible signal. 
%In order to produce Compressed Measurements, one first need to know what is the family of functions in which 
%the signal of interest is sparse. Depending on the case, one might be lucky and know that the signal is sparse in 
%a basis found in harmonic analysis (2.1) or one may have to spend some work in devising what these sparse basis is through 
%an algorithm dedicated to finding sparse dictionaries from a set of signal examples(2.2 and 2.3). 

\begin{align*}
x \approx f\left(x\right)
\end{align*}


When a signal is meant bo be sparse it is either representable via a small number of combined terms or series of many terms where the majority of coefficients is
close to zero.

\begin{align*}
x \approx b_{1} + b_{12} + b_{312}\\
 \text{v.s.}\\ x \approx 3b_{1} + 0.001b_{2} ... + 2b_{12} + 0.021b_{13} .. + 10b_{312}
\end{align*}



Simplification of signals in signal processing.
Those basis signal can be intepreted as atoms in a dictionary...

We define an atom elemtary signal or the smallest unit in a system/dict.
We define a dictionary as a set of diffent atoms.
Orthonormal (feature) vectors are othogonal and have unit length.


As a matter of fact a good basis or dictionary is essential for such a good signal approximation. \cite{} 
There are essential two major distinct ways to construct the desired dictionaries. First the construction of model based on basis functions that can represent our signal in
a sparse way  and second via algorithms that learn a sparse dictionary via a training process from a set of training data/signal samples.
The later approach of training good dictionaries has become a major task in the last decades. \cite{Mairal2010}





\subsection{History}

%Compressed sensing
The process of signal transformation that far back in the early 60s.\cite{Rubinstein2010}
FFT in 65s

%\begin{align*}
%\end{align*}

%\subsection{Discrete signals}
Rather then using continuous signals we concentrate on discrete signal representation.
Continuous signals would be better for .... . But our signal (image etc.) will be discrete anyway because of their initial digital representation. 
Besides the problem of coding becomes different in the continues space \cite{} Because of ...

\paragraph{Discrete consine transformation}
%\begin{align*}
%\end{align*}

\paragraph{Discrete wavelet transformation}
%\begin{align*}
%\end{align*}

approximate signals via combination of limited signal samples
Why?
signal analysis, compression, de-noise etc.

The early approaches in the 60s used combinations of cosine transformations. Coming from a continues representation this makes 
actually sense. The signal could be represented via ....

In 80s the search for better transformation basis became a major role in signal representation. \cite{}



\subsection{Splitting the problem}
\cite{Rubinstein2010}
In the last 15 years the concept emerged to interpret basis transforms as a set of signal atoms in a dictionary and the signals 
that they reconstruct as sparse linear combination of these atoms.
see \cite{Olshausen1996} and \cite{Mallat1993}
The benefit of this approach is that you can decouple signal coding and dictionary design and split the whole process of signal analysis into two tasks.
Coding of signals and the design of dictionaries.
Separating the problem into two distinct problems made the search for efficient coding of signals and construction of task specific dictionaries more flexible \cite{?}.

\begin{align*}
 x = D\alpha  x \approx D\alpha
\end{align*}


Size of database/dictionary
Quality

%In order to find 
%1. coding signals 
%2. find a good dictionary

%analog problem video compression but with less correlation between images and still image 

%Convergence of the dictionary learning
%Quality increase with size increase 
%Comparison with jpeg,jpeg2000 via RMSE


\section{Goal of this thesis}

The main purpose of the thesis is to
%The thesis concentrates on ...

\begin{itemize}
 \item Fast coding of many small signals under differnt constraints
 \item Efficient learning redundant dictionaries for large image databases from a large set of training data
 \item Evaluation of the quality and usage of the learned dictionaries
\end{itemize}

How many atoms do we need  for a ``good'' sparse representation of large image databases?

Evaluate the quality and size of learned big redundant dictionaries for optimal sparse coding of large image databases.
Experiments with small sets specific images like from sketches, still images of animations from disney and art styles like post-impressionistic images from Vicent van Gogh.
Search for a universal dictionary for databases of hundredths of thousands of images.

Evaluation of clustering of learning algorithm.
Convergence of different sizes and configurations of clustered learned dictionaries. 

Application of sparse coding for image compression and observation of problems (encoding time, quality benefit, etc.)
Comparission of compression with discrete cosine transformation approaches in jpeg and discrete wavelets transformation of jepg2000 and possible usage as an image descriptor.






%\Todo{OPTIONAL add structure for speed and model improvements Multi-scale, multi-channel, hierarchy}


