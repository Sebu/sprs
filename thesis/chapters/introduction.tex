\chapter{Introduction}
\label{sec:introduction}
%\thispagestyle{empty}
\section{Learning dictionaries}


 % for large image\\ databases}
\Todo{write at the end, merge with overview?}
%Signal <> is a topic <> since the 1960s. The basic theory is <>. Over the last
%few decades it has been
%a major <> to develop and find new bases for the signals. In the last 15 years 
%\Todo{get the reader}
Compressed sensing bla bla ...
Data networks, data transfer, error correction. For example experiments by
\cite{} have shown that just the right basis directly learned from images can
yield much better results in reconstruction than those based on mathematical
functions such as cosine transforms or wavelet\cite{Elad2006,Mairal2010}
transforms. These learned basis elements can lead to better reconstruction
quality of images in large image databases and assist tasks such as
classification, noise removal, image enhancement, organization and compression. 

Learning such basis is .. .often limited ... . In 2009 Mairal et
al.\cite{Mairal2009} presented an online learning algorithm that consumes 
rather than just a fixed batch. As an application example we will show
compression. We evaluation sparse coding and basis learning algorithms and
strategies for large image databases and transport recent discoveries in sparse
coding and basis learning into a real world example of image compression.
%of large image databases.

\section{Structure of thesis}
The next chapter gives an overview on the idea of sparse signal
representation with a short excurs into the history of signal analysis.
Chapter three describes the fundamental theory of sparse coding, introduces the
most common algorithms and presents usual applications of sparse coding. The
fourth chapter introduces algorithms for the design of over-complete
dictionaries of basis elements and presents current related work in the field of
compressed sensing. The fifth chapter describes the applied methods for sparse
coding, the generation of dictionaries via clustered training with a large image
database of 1,000,000 images collected from \url{flickr.com}. Presents
experiments regarding the structure of dictionary elements, convergence of the
training process, comparison of signal representation quality and adaptability
of sparse coding for image compression. The two final chapters present the
results of the experiments, discuss these results and new findings and show
potential improvements in future work. 

\paragraph{Notation}
$a[i]$ element i of a vector
alpha,X,x,D,d,L,epsilon error, L,lambda , A,I sets


