\chapter{Introduction}
\label{sec:introduction}
%\thispagestyle{empty}
\section{Learning dictionaries for large image databases}

\Todo{write at the end, merge with overview?}

%Signal <> is a topic <> since the 1960s. The basic theory is <>. Over the last few decades it has been
%a major <> to develop and find new bases for the signals. In the last 15 years 

%\Todo{get the reader}
Experiments by \cite{} have shown \cite{} that image bases from natural images can yield much better results in compression/reconstruction than
those based on mathematical functions e.g. cosine and wavelets.
\cite{Mairal2010}
\Todo{esential problem/idea}
Transport recent discoveries in sparse coding learned overcomplete dictionaries into compression/organisation
of large image databases.
These dictionaries can lead to better compression quality of large image databases and assist other tasks (classification/de-noising/image enhancement/organisation ...)


\section{Structure of thesis}
Chapter two gives a  introduction into idea of sparse signal representation with a short excurse into the history of signal analysis.
Chapter three describes the fundamentel theory of sparse coding, introduces the most common algorithms and presents usual applications of sparse coding.
The fourth chapter introduces algorithms for the design of overcomplete dictionaries and presents current related work in the field of compressed sensing.
The fifth chapter describes methods for sparse coding, the generation of dictionaries via clustered training with a large image database 
of 1.000.000 images collected from \url{flickr.com} and presents experiments regarding structure of dictionary elements, 
convergence of the training process, comparission of signal representation quality and adaptability of sparse coding for image compression.
The two final chapters show the results of the experiments, final conclusions and discussion of future work. 

