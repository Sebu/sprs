\chapter{Introduction}
\label{sec:introduction}
%\thispagestyle{empty}
\section{Learning dictionaries}

Finding the most significant basis elements, that make up an
signal, is a popular task in signal processing. A sparse selection of such
elements coded together in the right way can assist in imaging tasks such as
noise removal, image enhancement, classification or compression. A selection
process also known as sparse coding.

Experiments \cite{Chen1998,Elad2006} have shown that dictionaries of 
elements directly learned from images can yield much better results in
such sparse coding tasks than those designed from mathematical functions like
cosine or wavelets. But most of the current learning algorithms are limited to
small batches of only 100.000 to 1.000.000 training samples. Learning from
larger sets of millions of samples found in large images databases is a
challenge. In 2009 Mairal et al.\cite{Mairal2009} presented an new online
learning algorithm that can constantly consume new samples rather than just a
fixed batch. 

We want investigate this algorithm on how we can learn dictionaries
for large image databases and how we can effectively code the elements.
As an example application to use these dictionaries we develop a sparse coding
compression algorithm and compare the performance to other
compression algorithms.

\section{Structure of thesis}
The next chapter gives an overview on the idea of sparse signal
representation with a short excursus into the history of signal analysis.
Chapter three describes the fundamental theory of sparse coding, introduces the
most common algorithms and presents usual applications of sparse coding. The
fourth chapter introduces algorithms for learning over-complete
dictionaries of basis elements and presents current related work in this field.
The fifth chapter describes the applied methods for sparse
coding, the generation of dictionaries via clustered training with a large image
database of 100,000 to 1,000,000 natural images collected from \url{flickr.com}.
Presents experiments regarding the structure of dictionary elements, convergence
of the training process, comparison of signal representation quality and
adaptability of sparse coding for image compression. The two final chapters
present the results of the experiments, discuss these results and new findings
and show potential improvements in future work. 

\section*{Notation}
\begin{tabular}{c l}
$x$ & scalar\\
$T$ & constant\\
$\vec{x}$ & vector\\
$\vec{1}$ & vector with all elements 1\\
$\vec{x}[i],x_i$ & $i$-th element of vector $\vec{x}$\\
$\mat{M}$ & matrix\\
$A$ & active set\\
$A^C$ & complment of set $A$ in relation to set of all indices\\ 
$\vec{x}_A$ & vector $\vec{x}$ reduced to rows of set $A$ \\
$\mat{M}_A$ & matrix $\mat{M}$ reduced to rows and columns of set $A$ \\

\end{tabular}



