\chapter{Introduction}
\label{sec:introduction}
%\thispagestyle{empty}
\section{Learning dictionaries} %for large image databases}

\Todo{write at the end, merge with overview?}

%Signal <> is a topic <> since the 1960s. The basic theory is <>. Over the last few decades it has been
%a major <> to develop and find new bases for the signals. In the last 15 years 

%\Todo{get the reader}
Experiments by ... have shown that image bases from natural
images can yield much better results in compression/reconstruction than
those based on mathematical functions e.g. cosine and
wavelets.\cite{Elad2006,Mairal2010}
\Todo{essential problem/idea}
Transport recent discoveries in sparse coding learned over-complete
dictionaries into compression/organization of large image databases.
These dictionaries can lead to better compression quality of large
image databases and assist other tasks (classification/de-noising/image
enhancement/organization ...)


\section{Structure of thesis}
The next chapter gives an overview on the idea of sparse signal
representation with a short excursus into the history of signal analysis.
Chapter 
three describes the fundamental theory of sparse coding, introduces the most
common algorithms and presents usual applications of sparse coding. The fourth
chapter introduces algorithms for the design of over-complete dictionaries and
presents current related work in the field of compressed sensing. The fifth
chapter describes the applied methods for sparse coding, the generation of
dictionaries via clustered training with a large image database of 1.000.000
images collected from \url{flickr.com} and presents experiments regarding
structure of dictionary elements, convergence of the training process,
comparison of signal representation quality and adaptability of sparse coding
for image compression. The two final chapters show the results of the
experiments, discussion findings conclusions and potential improvements in
future work. 

