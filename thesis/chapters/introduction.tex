\chapter{Introduction}
\label{sec:introduction}
\section{learning dictionaries for large image databases}
Signal <> is a topic <> since the 1960s. The basic theory is <>. Over the last few decades it has been
a major <> to develop and find new bases for the signals. In the last 15 years 

\Todo{motivation, get the reader}

Experiments by \cite{} have shown \cite{} that image bases from natural images can yield much better results in compression/reconstruction than
those based on mathematical functions e.g. cosine and wavelets.


\cite{Mairal2010}

\Todo{esential idea}
Transport recent discoveries in sparse coding / learned overcomplete dictionaries into compression/organisation
of large image databases.

These dictionaries can lead to better compression quality of large image databases and assist other tasks (classification/browsing/search in large image databases)

\section{structure of thesis}
Fundamentel introduction into sparse coding and dictionaries, their generation and related work,
Method of generation and implementation of it, test with large image database of .....000 images collected from ..... .
Application to large image databases (also adaptive growing). Final conclusion discussion of future work. 


