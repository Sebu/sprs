\chapter{Dictionaries}
\thispagestyle{empty}

\section{Structure}
Set of orthonormal signal vectors.
\subsection{Designed}


\begin{description}
 \item[cosine]
 \item[wavelets]
 \item[curvelets/contourlets/bandelets]
\end{description}

\subsection{Learned}
Recent research has shown that learned dictionaries show better compression quality than small analytic dictionaries \cite{Aharon2006} \cite{Chen1998} 
Rather than selecting an appropriate mathematical construct that can reconstruct a certain group of signals close enough the data, a subset or similar data
is used as a training set to generate the dictionary itself.

The atoms are not orthonormal. The dictionary is overcomplete. Overcomplete ...


%\subsubsection{k-svd}
\section{Learning}
In the last decade several learning algorithms have been proposed.
The basic concept of following algorithms for learning redundant dictionaries is to alter a initial start dictionary
in this way that it can sparsely reconstruct a set of training data with minimal error. 

\subsection{Batch}
MOD \cite{Engan1999}
Generalized PCA \cite{}
ILS-DLA \cite{}
K-SVD \cite{Aharon2006}


\subsection{Online}
In 2010 new training algorithms were presented that enabled on-line learning of dictionaries. 
In contrast to batch learning algorithms with a fixed training set the new approaches enabled
This is good for very large training sets or training data that is unknown at the start of the training process.

... RLS-DLA \cite{Engan2010} ...

\Todo{find the Japanese one}

In 2010 Mairal et. al. \cite{Mairal2010} proposed an on-line approach which from now on will be called TrainDL

\subsection{TrainDL}
\label{sec:mairal}
The algorithm presented by Mairal et al. \cite{Mairal2010} is ...

\begin{algorithm}
\caption{Online dictionary learning \cite{Mairal2010}}
\begin{algorithmic}[1]
\REQUIRE $x \in \mathbb{R}^m,  p \left( x \right), \lambda \in \mathbb{R}, D_0 \in \mathbb{R}^{m \times p}, T$
\STATE $A_0 \in \mathbb{R}^{p \times p} \gets  0, B_0 \in \mathbb{R}^{m \times p}\gets 0$
\FOR {$t = 1$ to $T$}
\STATE Draw $x$ from p(x).
\STATE Sparse code:
\begin{align*} 
\alpha_t \equiv \argmin_{\alpha\in\mathbb{R}^{p}}  \lVert x_t - D_t\alpha \rVert^{2}_{2}  +  \lambda \lVert \alpha \rVert_{1}
\end{align*}

\STATE $A_t \gets A_{t-1} + \alpha_t\alpha_t^T$
\STATE $B_t \gets B_{t-1} + x_t\alpha_t^T$
\STATE Compute $D_t$ using \prettyref{alg:update}, with $D_{t-1}$ as warm restart 
\begin{align} 
D_t \equiv \argmin_{D \in \mathbb{R}^{m x k}}  \frac{1}{t} \sum_{i=1}^t \left( \lVert x_i - D\alpha_i \rVert^{2}_{2}  +  \lambda \lVert \alpha_i \rVert_{1} \right) \label{eq:update}
\end{align}
\ENDFOR
\RETURN $D_T$
\end{algorithmic}
\end{algorithm}


\begin{algorithm}
\caption{Dictionary Update}
\label{alg:update}
\begin{algorithmic}[1]
\REQUIRE $D=[d_1,...,d_p] \in \mathbb{R}^{m \times p}, A=[a_1,...,a_p] \in \mathbb{R}^{p \times p}, B=[b_1,...,b_p] \in \mathbb{R}^{m \times p}$
\REPEAT
\FOR {$j = 1$ to $p$}
\STATE update j-th column to optimze for \prettyref{eq:update}:
\begin{align*}
u_j \gets \frac{1}{A[j,j]}\left(b_j-Da_j\right)+d_j \\
d_j \gets \frac{1}{\max\left(\lVert u_j \rVert_2,1\right)} u_j
\end{align*}

\ENDFOR
\UNTIL convergence 
\RETURN $D$
\end{algorithmic}
\end{algorithm}


\section{Learning for the Task}
It has been shown that learning basis specific for certain tasks can lead to the best results\cite{sapiro?}.  <>
Based on this discovery we will concentrate on a specific class. 
%<> Join the basis for natural images and cartoon/line images.
We will also concentrate on real practical data. This means typically 3-channel data of 1+ mega-pixels images found on image hosting services like flickr, twitpic, imgurl, picasa etc.


\section{Related work}
%Current research is primary concentrating on other tasks. 
%Like ... related work :)
\begin{description}

\item[texture synthesis]  

\item[compression of facial images] 
In 2008 Bryt and Elad \cite{Bryt2008} 

\item[training with a neural network]
Learning Multiple Layers of Features from Tiny Images \cite{Krizhevsky2009}


\item[multi-scale]
Rather than encoding uniform segments of a bigger signal this technique encodes different sized 
blocks (multiple times of the default block size). The idea is that aligned smooth regions with low variance can be 
can be combined an encoded as one signal rather than e.g. 4. \cite{saprio}
\Todo{picture with different scales}

\item[multi-channel]
When dealing with multi-channel data (e.g. RGB images) 
each channel can be coded separate or as a concatenation of each signal into a single one.
These approaches can lead to color bleeding \cite{mairal08sparse} \Todo{add pic}
Fortunately this problem vanishes with big dictionaries and large training sets. \cite{mairal08sparse}
Nevertheless a solution would be to add a constraint to the sparse coding that pays attention the correlation of the channels.
\Todo{add formula from paper}

\item[double sparsity]
Rather then directly using a collection of trained atoms as dictionary elements. This approach uses sparse coding
to encode the signals of a bigger dictionary with the help of a smaller on. \cite{double sparsity} 
\Todo{formula $B$ .. basis, $A$ sparse matrix, $D=BA$ ...}

\item[group sparsity]
similar patches should admit similar patterns
\cite{group sparsity}
\Todo{picture with ditribution difference}

\item[hierarchy]
Adding hierarchy to dictionaries can lead to faster coding or use in structure analysis ( text documents )
\cite{Jenatton2010}
\Todo{picture with hirarchy }



\end{description}

