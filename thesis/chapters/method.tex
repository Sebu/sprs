\chapter{Method}
%\thispagestyle{empty}


\section{Performance comparison}
\Todo{move to results? or at end of method}

Before we can present the results of the experiments it is necessary to
introduce some metrics to measure their performance.


\paragraph{Root mean squared error (RMSE)} is error measurement that ...

Comparing the a signal $x$ with a reconstruction $y$ (noisy approximation) of
it. The root mean squared error is defined as:
\begin{align}
 RMSE = \left[\frac{1}{n} \sum_{i=0}^{n}{\lVert x_i - y_i\rVert^{2}}
\right]^{\frac{1}{2}}
\end{align}
It is primarily used for ...

\paragraph{peak signal-to-noise ration (PSNR)}

\Todo{Normalized version of RMSE}

\begin{align}
 PSNR = 20 \cdot \log_{10} \left(\frac{MAX}{RMSE}\right)
\end{align}
Where $MAX$ is the maximum possible value of our signal. For an 8-bit image it
would be 255. For a 32-bit normalized image it would be 1. The PSRN is expressed
in a logarithmic decibel scale.

Typical values lossy reconstruction are between 20dB and 50dB.


%e.g. relevant for de-noise


\paragraph{Bits per pixel (bpp)}
For comparison of compression ratio of images a well known practice is to
measure the bits requires for a single pixel. For example an uncompressed RGB
color image with 8-bit of color depth requires 24-bits per pixel, respectively
8-bit for a single channel gray scale image.

Looking at other well known compression algorithms such as JPEG or JPEG 2000
a common ratio is about ~1.8 bits-per-pixel for average quality
compression(Q=50) of a natural image. \Todo{example image?}

Besides the actual pixel data there is certain amount of extra data from file
headers. 
%The sparse vector for every are encoded in 2 bytes index and a single byte
quantized coefficient. Additional zero coefficients from the quantization step
are removed.

\paragraph{Test data}
In addition to the metrics it is common practice to use some image sets to
compare test results of the different algorithms and parameter configurations.
For comparison of the reconstruction quality and compression quality we use a
well known set of images from the USC-SIPI Image Database often used in image
processing for evaulation of algorithms. Including pictures such as Lena,
Mandril and Peppers.

\section{Signal representation}
\label{sec:signal_representation}
As we use natural color images and graphics we require to use multi-channel
data. Common 3-channel from a RGB color space. %The initial signal data 
We use an approach to transform our initial signal into other color spaces like
in jpeg encoding to account natural human reception of color data. Experiment
....

%\subsection{color/signal representation}

\section{Sparse coding step}
\subsection{Batch-OMP}
In 2008 Rubinstein et al.\cite{Rubinstein2008} presented the batch-OMP. The
\prettyref{alg:batchOMP} is a modified version of the
orthogonal-matching-pursuit (\ref{sec:omp}) that can effectively sparse codes
multiple signals in parallel. The key of the optimization is the pre-computation
of the gram matrix $G=D^TD$ and a cholesky factorization.

\subsubsection*{Optimizations}
While $L$ is very small the main factor of the computation is
the multiplication of $D$ with a vector with a complexity of $O(mp)$. 
This modification reduces the complexity of the OMP from $O\left(L2mp + 2L^2m +
2L(p+m) + L^3\right)$ to $O\left(2mp + L^2p + 3Lp + L^3\right)$ 
with pre-computation $\left(mp^2\right)$ as shown in\cite{Rubinstein2008}. 


\Todo{rewrite with batch-OMP}
\begin{algorithm}
\caption{multi signal optimized OMP}
\label{alg:batchOMP}
\begin{algorithmic}[1]
\REQUIRE $X =[x_1,...,x_k]  \in \mathbb{R}^{m \times k}, D  =[d_1,...,d_p]  \in
\mathbb{R}^{m\times p}, \epsilon \in \mathbb{R}$
\STATE $L\gets[1]$
\WHILE {$\epsilon^{n-1}> \epsilon $}
\STATE
\IF {$n>1$}
\STATE
\begin{align}
L \gets \left[
\begin{array}{ccc}
L & 0\\
w^T & \sqrt{1-w^Tw}
\end{array}
\right]
\end{align}
\ENDIF
\STATE $I \gets I \cup \{ k\}$
\STATE
\STATE $n \gets n+1$
\ENDWHILE
\RETURN $\alpha$
\end{algorithmic}
\end{algorithm}

\begin{algorithm}
\caption{multi signal optimized OMP}
\label{alg:batchOMP}
\begin{algorithmic}[1]
\REQUIRE $X =[x_1,...,x_k]  \in \mathbb{R}^{m \times k}, D  =[d_1,...,d_p]  \in \mathbb{R}^{m\times p}, \epsilon \in \mathbb{R}$
\STATE pre-compute gram matrix: $G=D^TD$
\FOR[all signals in parallel] {$j = 1$ to $k$ }
\STATE $\alpha_j \gets 0, r_j \gets x_j $ (residual) $, A_j=\emptyset$
\FOR {$i = 1$ to $L$}
\STATE Select atom with maximum correlation with residual: 
\begin{equation*}
i \gets \argmax_{i \in A_j^C} \lvert \left<d_i,r_j\right> \rvert
\end{equation*}
\STATE solve with cholesky: $G_{inv} \gets G_A^{-1}$
\STATE update active set: $A_j \gets A_j \cup \{i\} $
\STATE update residual: $r_j \gets \left(I-D_AG_{inv}D_A^T \right)x_j$
\STATE update coefficients: $a_{j_A} \gets G_{inv} D_A^T x_j $
\ENDFOR 
\ENDFOR 
\RETURN $\alpha$
\end{algorithmic}
\end{algorithm}


\subsection{LARS-Lasso}
\subsubsection*{Optimizations}
The same pre-computation of the gram matrix used in the batch-OMP can be
also applied in the LARS-Lasso algorithm. Reduction the complexity. %to:
\Todo{lasso complexity}

The complete regularization path based on\cite{Efron2004}


\section{Training step}
Keeping the training sparse vectors too sparse can lead to dictionaries with
only low frequency elements. Using LARS-Lasso for training generates a good
dictionary for LARS and OMP. Training with OMP leads to a dictionary more with a
lot of noise that leads better coding results when using OMP 
% is LARS better?
As described in\ref{sec:signal_representation} the learning process is not
limited to grayscale oder RGB image signals. The presented algorithm can be
applied to diffrent kinds of signals. Such as images in other color
spaces, low/high pass filtered sub-images or signals from another domain like
audio or text. 
%Some of them are used in our compressions experiments.


\subsection{\trainDL}
A more in-depth look at the Mairal on-line training algorithm \ref{sec:mairal}. 
Why use this algorithm? Robust, convergent, fast, on-line

\subsection{Initialization}
%\subsection{Dictionary initialization}
%\paragraph{Random data} 
%\paragraph{Random elements from training set}
At the start of the training process it is requires to initial the
dictionary with start data. Otherwise the sparse coding step will only find
trivial solutions with all coefficient being zero. Which then will have no
effect in the learning process.

There are two common ways to initialize the dictionary using random data or
selecting random elements from the training data. Both ways will be tested and 
the quality will be compared.

\subsection{Clustering}

The cluster merge of results -> 
it is a map/reduce approach. 


\subsubsection*{Optimizations}
%\subsection{Adaptive}
%Thanks to the on-line approach of the training step an adaptive approach can be
%used. 

\section{Training sets}
%\subsection{Learning specific dictionaries}
As mentioned earlier\ref{sec:learnForTheTask} one of key elements 
of the learning algorithms is the learning of specialization dictionaries.
Task specific training data is a common way to solve the problem of finding 
the right dictionary. Here learning for the task comes into
account. Such as de-noising/in-painting dictionaries directly learned from the
initial signal that gets de-noised or restored from in-painting. If the task
gets bigger it sounds logical to increase the size of training data and take a
bigger variety of signals to learn from.  We have a closer look at differences
of learned elements from specific sets of images. These sets include sketches,
still images of animations from Disney and post-impressionistic images from
Vicent van Gogh. 
Are big image collections specific enough to benefit from 



\section{Image compression}
Todays lossy image compression algorithms are primary based on scientific
findings about visual perception reaching far back in the 1970s\cite{?} and
good entropy encoding of the data. Both elements led to to the
following main steps in lossy image compression algorithms:
\begin{itemize}
 \item lossy color space conversion and separation
 \item spatial separation, sub-sampling of images
 \item coding/transformation of image signals
 \item quantization to reduce of the number of non-zero coefficient 
 \item lossless entropy coding of the coefficient 
\end{itemize}
%Copy
%The human eye is good at seeing small differences in brightness over
%a relatively large area, but not so good at distinguishing the exact strength
%of a high frequency brightness variation. This allows one to greatly reduce the
%amount of information in the high frequency components. This is done by simply
%dividing each component in the frequency domain by a constant for that
%component, and then rounding to the nearest integer. This rounding operation is
%the only lossy operation in the whole process if the DCT computation is
%performed with sufficiently high precision. As a result of this, it is
%typically the case that many of the higher frequency components are rounded to
%zero, and many of the rest become small positive or negative numbers, which
%take many fewer bits to represent.

\paragraph{Signal conversion}
The human eye is good at sensing small difference in brightness but lacks the
ability exactly differentiate small changes in color. Converting an image from a
typical three channel $RGB$ color space to $YC_bC_r$ color space takes advantage
of
this effect. The YCbCr color space splits an image
into a brightness/luminance (Y) and two color/chroma components (Cb
and Cr). The visual more relevant brightness/luminance channel will be coded
in the full resolution while the color/chroma components can be sub-samples
with only minor loss in perceptual quality. This procedure is called
\emph{chroma subsampling} and is one key element to of preserving perceptual
quality in lossy image compression while.
%color space, high/low pass filtering

\paragraph{Signal coding}
The actual coding step of the prepared signas is a simple application of one
of the sparse coding algorithms presented in\ref{chap:sparse_coding}.
But before we can actualy code signals we need a good dictionary.
We learn dictionaries from our training sets 
block sizes
Number of coefficients


\paragraph{Quantization}
%copy
The human eye is good at seeing small differences in brightness over a
relatively large area, but not so good at distinguishing the exact strength of a
high frequency brightness variation. This allows one to greatly reduce the
amount of information in the high frequency components. 

Fixed quantization or learned from dictionary

%The JPEG algorithm 
We adopt the idea of visual perception idea of JPEG to our
approach and sort our element by their frequencies.
The random distribution of the atoms in the dictionary 
We concentrated on two fixed quantization factor and the analysis of the
dictionary elements.



\paragraph{Data encoding}

Coefficients
Indices
In 1999 Lewicki et al.\cite{Lewicki1999} already made some on comparison 
the bpp required for encoding of the sparse matrix data. Besides the estimation
of the compression we also really applied some coding schemes to compare real
world compression data.
\cite{Murray2006}

\paragraph{Run-length encoding (RLE)}
\paragraph{Huffman coding}
\paragraph{Arithmetic coding} is ...
While arithmetic coding often yields better compression results than huffman
coding. The higher processing requirements for coding and the current patent
situation makes the algorithm unpopular for coding compared to the archived
compression benefit.

We apply RLE and huffman to our sparse matrices. 
%The sparse vector for every are encoded in 2 bytes index and a single byte quantized coefficient. Additional zero coefficients from the quantization step are removed.


As mentioned in in \ref{sec:headers} we are aware of the fact that besides
the actual sparse matrix data we also have other file informations and meta
data but .  



\section{Implementation}
%\lstinputlisting[language=C++,caption=Training]{listings/test.cpp}

\subsection*{Hardware setup}
%Problems
%Big matrices
%Speed 
%Compression
%The 
Computations were made on Apple MacPro with ... \Todo{mac config}
and a cluster consisting of 100 nodes with each AMD Athlon X2 \Todo{athlon
config} running Debian 32-bit. Each client in the cluster calculated batches of
100 till 10,000 images based on the size of the trainig sets.

\subsection*{Software}

\Todo{add diagram}
Split image into sub images, convert sub image to samples from image blocks, 
train dict with samples or just code samples - save dict/image

The whole test suit is written in C++ with the help  of OpenCV, Eigen and OpenMP libs.

Eigen\cite{Eigen} is a template library for fast vector and matrix operations 
and includes some linear algebra algorithms. It is mainly used for operations on
dense and sparse matrices and solving of linear equation systems.

OpenCV\cite{OpenCV} as in Open Computer Vision Library is mainly used for
image read and write operations and color space conversion. OpenMP
\footnote{www.openmp.org\cite{OpenMP}} is a preprocessor based application
programming interface (API) for C,C++ and Fortran that enables the user to
distribute code block and loops to multiple CPU cores. In situations where it is
applicable OpenMP is used to utilize multi-core CPUs. 

\paragraph{Coding}
The OMP implementation is based on the batch-OMP implementation
from\cite{Rubinstein} The implementation of the LARS-Lasso algorithm is based on
the implementation of\cite{Strand2005} and the original paper\cite{Efron2004}

\paragraph{Trainig}
The \trainDL trainer is a straight forward implementation of the
algorithm presented in\cite{Mairal2010}. It is an implementation of the basic
version of the algorithm with the batch optimization applied. Both coding
methods (OMP and LARS-Lasso) can be utilized for the coding step.

\paragraph{Misc}
\Todo{Other: imagemagick for conversion of test image -> JPEG,JPEG 2000 and
comparison of images} Besides the actual coding software some other tools for
additional tasks were used.

Several bash and ruby scripts for workload distribution onto the
clients, aggregation and evaluation of the results. The imagemagick toolkit for
image conversion into JPEG and JPEG 2000 and for some image comparison tasks.






\Todo{add diagram}


