\chapter{Method}


\section{Signal coding/representation}
As we use natural color images and graphics we require to use muti-channel data. Common 3-channel from a RGB color space.
We take an approach to transform our initial signal into other color spaces like in jpeg encoding to account natural human reception of
color data. 
Expriment ....
\subsection{color/signal representation}

\section{coding step}
\subsection{batch-OMP}
A modified version of the OMP\footnote{\nameref{sec:omp}} that can effectively sparse codes multiple signals in parallel.

\subsection{LARS-Lasso}
The complete regularization path based on This \cite{lars.m}

\section{training step}
\subsection{Mairal2010}
A more indepth look at the mairal online training algorithm. 
Why use this algorithm? Robust, convergent, fast, online

\subsection{adaptive}

\subsection{quality messure}
PSNR, mean square error MSE, adaptive training 

\section{Encoding/Compression}
see JPEG: quantization, RLE or entropy

\subsection{Quantization}


\section{Implementation}
\lstinputlisting[language=C++,caption=Training]{listings/test.cpp}
The whole test suit is written in C++.
Eigen lib for vector and matrix operations and some linear algebra algorithms.
OpenCV for image read/write and color space conversion.
In situtations where ist is applicable we use OpenMP to make use of muti-core CPUs. 

Coder
The OMP implementation is based on the OMP implementation from \cite{OMPBox}
The implementation of the LARS-Lasso algorithm is based on the implementation of \cite{lars.m}

Trainer
The Mairal2010 trainer is a straight forward implementation of the basic version of the algorithm presented in \cite{Mairal2010}.
With the batch optimization.


