\chapter{Conclusion and Future Work}

\section{Summary}
Over the course of this thesis we build a framework to sparse code and learn
dictionaries from big sets of different samples as found in large image
databases. We implemented the Batch-OMP and the LARS-Lasso sparse coding
algorithms for $\ell_0$ and $\ell_1$ regularization and the ODL by Mairal et
al. for learning dictionaries. 
In addition to this we added a cluster learning and merging strategy to the
framework to speed up learning of large dictionaries. We proposed a simple
sparse coding based compression algorithm that can compete with and surpass JPEG
compression in certain situations.


\section{Conclusion}
The right regularization and the the right training data
play a big role for learning the right dictionaries and gaining
good reconstruction quality. 

It turn out that that the $\ell_1$ regularization of LARS-Lasso has a selection
strategy which is very good at learning dictionary atoms.  And also good for
reconstruction tasks. Using LARS-Lasso for training generates a good dictionary
for LARS and OMP sparse coding tasks.
While the $\ell_0$ regularization of the OMP, with its greedy nature, has a very
aggressive selection strategy.  This very noisy selection strategy is less
practical than LARS for the ODL learning step. Such as a high number of learning
coefficients and bad initialization. 


Using bit image sets for learning can lead to very universal dictionaries. If
the images fit into the same category it is insignificant which images you
chose. The results will not much differ. When extracting samples from JPEG
images for learning keep in mind that we can only learn what the data provides.
Samples from JPEG especially non-overlapping $8 \times 8$ blocks will help to
learn DCT atoms.

%or want to verify if it is a JPEG image.
%Dictionaries to universal ... improved results to DCT .. but varying with field
%of application. possible no universal solution but a good way for better
%understanding of the key elements
% Better understanding the selection strategies of the algorithm and their
% meaning for perceptional image quality. And evolution of the structure of
% learned dictionaries.
% Similar to natural images.
% compression dicts

Dictionaries can be made much bigger than usual but the increased quality
comes with the price of computational complexity. Clustering learning strategies
can help to learn more indipendent and still lead to equal reconstruction
quality as single big dictionaries.

When it comes to compression tasks the greedy OMP performs better
than the LARS-Lasso. As it sheers for less coefficients. Sparse coding 
compression algorithms can lead to similar compression quality as JPEG 
for low bit rates. And they can surpasses it in certain situations. But
currently a lot of tweaking to the quantization step is required. 

Besides that JPEG and JPEG 2000 are highly optimized lossy image compression
algorithms. Years of research went into observation human vision. Especially the
quantization and and the entropy encoding step are highly optimized for specific
data. It is hard to compete with JPEG, the market leader in this field.  Even
JPEG 2000, which leads to better results than JPEG especially in the area block
artifacts, did not catch up with the success of JPEG. The reason lies in the
potential patent issues and higher computational requirements of the algorithm
compared to the quality benefits and the wide acceptance of JPEG.

Our experiments are just a first glimpse of that can be done to use sparse
coding and dictionary learning for large image databases.

\section{Future work}
The amount of publications regarding sparse coding and machine learning in the
last decade indicate strong ongoing research in this field. Especially
brining structure to the learning stage is one of the most active topics.

\paragraph{Structure}
The results gained on the appearance of the atoms can be used to think
about more advanced ways to structure or generate the atoms.

\subsection{Different training sets}
One of our findings is the fact that we tend to learn DCT atoms from natural
JPEG images. Testing with larger sets of images that did not go through a JPEG
compression is one major topic to address in the future. Raw camera images or
tiff images are a possible source. 

\subsection{Framework improvements}
To make further and more flexible experiments. The framework needs be be
extended in terms of speed and new dictionary learning strategies.

\paragraph{Speed}
While the implementation of the Batch-OMP and the ODL are already very fast
the LARS-Lasso implementation is quiet slow compared to the KLT implementation
in the SPAMS framework. In the near future the SPAMS framework will provide a
C++ interface\footnote{\url{http://www.di.ens.fr/willow/SPAMS/faq.html}}. Adding
support for this interface could lead to speed improvements. 

\paragraph{Clustering}
Also taking the clustering strategy two even further is to learn smaller
dictionaries from different sets of images or characteristics. Such as
paintings, gray-scale sketches and smooth animation images. For example
learning dictionaries for very specific classes of images can be used for
classification tasks. Something that can come in handy when applying it to
operations like search in large image databases.


\subsection{Compression improvements}
\paragraph{Reducing block borders} The presented compression experiment
takes a very simple approach without taking correlation between adjacent
blocks into account. Similar to the JPEG block strategy. But other than
the JPEG basis transforms the learned atoms are not limited to locality in
the frequency domain. Splitting images in high and low pass sub images and
learning multi-scale dictionaries for each sub image could improve image
quality. The framework can be easily extended to learn and code different kinds
of sub-images.

\paragraph{Better entropy coding}
The coding of the coefficients already works quiet well. But the coding of the
indices is currently limited as the sparse coding algorithm tends to select a
lot of different atoms. A structured sparse coding approach as mentioned in
\prettyref{sec:sparse_related} could lead to less selected atoms and a better
entropy encoding. Also the required bits for each index nead further
optimization.









